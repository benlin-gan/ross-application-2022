\documentclass{article}
\title{Ross Application Problems 2022}
\date{2022-02-21}
\author{Benlin Gan}
\usepackage{amsmath}
\usepackage{amsfonts, amssymb}
\DeclareMathOperator{\lcm}{lcm}
\renewcommand{\thefootnote}{\fnsymbol{footnote}}
\usepackage{csvsimple, listings}
\begin{document}
  \maketitle
  \newpage
  \section{Problem 1: Powers of 2 in $\mathbb{Z}_m$}
  Note: Throughout this section I will use $\mathbb{N}$ to refer to the set of non-negative integers.
  \subsection{Playing with $\mathbb{Z}_m$}
  \begin{align*}
    (2^n) &= (2, 4, 6, 8, 16, 32, 64, 60, 52, 36, 4, 6, 8, ...)   \hspace{1em}  (\textrm{in } \mathbb{Z}_{68})\\
    (2^n) &= (2, 4, 6, 8, 16, 32, 64, 29, 58, 17, ..., 62, 25, 50, 1, 2, 4, 8 ...) \hspace{1em}     (\textrm{in } \mathbb{Z}_{99})\\
    (2^n) &= (2, 4, 6, 8, 16, 32, 64, 1, ...) \hspace{1em}     (\textrm{in } \mathbb{Z}_{127})\\
  \end{align*}
  \subsection{Part A: Proof of Periodicity}
  Assume there $\exists k, l \in \mathbb{N}$ such that $k \neq l$, and $2^k \equiv 2^l \mod m$. Then
  \begin{align}
    m \mid 2^k - 2^l 
  \end{align}
  Then since any multiple of a number is divisible by everything the original number was divisible by:
  \begin{align}
    m &\mid 2^i(2^k-2^l), \forall i \in \mathbb{N}\\
    2^{k+i} &\equiv 2^{l+i} \mod m, \forall i \in \mathbb{N}
  \end{align}
  Therefore once we find one repeated element in $(2^n)$, everything after it also repeats. Furthermore, we know that this repeated element $(2^k, 2^l)$ must exist, because there are only $m$ unique elements in $\mathbb{Z}_m$, but an infinite number of elements in $(2^n)$.
  \subsection{Part B: Tail Function}
  \subsubsection{Tail = 0}
  If $\rho (m) = 0$, then there exists a postive integer $k > 1$ such that $2^k \equiv 2^1 \mod m$. Then $m \mid 2^k - 2$. This simple manipulation also proves the converse, that $p(m) = 0$ for all $m$ where there exists a postive integer $k$, such that $2^k - 2$ is a multiple of $m$.\\

  Looking at the example given in the problem statment where $m=10$, we can see that $\rho(10) = 0$. We should be able to find a $k$ such that $2^k -2$ is a multiple of $m$, and indeed $k=5$ works.
  \subsubsection{Tail = 1}
  If $\rho (m) = 1$, then there exists a postive integer $k > 2$ such that $2^k \equiv 2^2 \mod m$, and $\forall j \in \mathbb{N}, j > 1 \mid 2^{j} \not\equiv 2^1 \mod m$. Then, using the logic in the last section, if $\rho(m) = 1$, we should be able to find a $k$ such that $2^k - 4$ is a multiple of $m$ but $2^j - 2$ is not.\\

  Looking at $\rho(28) = 1$, we can see that $2^5 - 4$ is a multiple of 28, and we know that $\forall j \in \mathbb{N}, j > 1, 28 \nmid 2^j - 2$, because $2^j$ will be divisible by 4, so $2^j - 2$ won't, so it also isn't divisible by 28. 
  \subsubsection{Tail = 2}
  if $\rho (m) = 2$ then there exists a postive integer $k > 3$ such that $2^k \equiv 2^3 \mod m$, and $2^{k-1} \not\equiv 2^2 \mod m$, and $2^{k-2} \not\equiv 2^1 \mod m$. So, if $\rho(m) = 2$, we should be able to find a $k$ such that $2^k - 8$ is a multiple of $m$, but $\forall j \in \mathbb{N}, j > 2, 2^{j} - 4$ and $\forall j \in \mathbb{N}, i > 1, 2^{i} - 2$ aren't.
  \subsubsection{Generalized Tail Finding}
  Given an arbitrary postive integer $m$, we know that $\rho(m)$ is the smallest postive integer $i$ such that $\exists k \in \mathbb{N}, 2^{i+1}(2^k - 1)$ is a multiple of $m$. Obviously, $2^k-1$ isn't divisible by 2, but given a high enough $k$ it will be divisible by the largest odd factor of $m$.\\

  We know this because of the periodic nature of the $(2^n)$ sequence. Given that $\exists a, \exists b, \forall m \in \mathbb{N}$, $2^{a+b} \equiv 2^b \mod 2m+1$, we can divide by $2^b$ to conclude that:
  \begin{align}
    2^a \equiv 1 \mod 2m+1 \label{pf}
  \end{align}
  Therefore the power of 2 in the prime factorization of $m$ must come solely from $2^{i+1}$, and conversely, if $2^{i+1}$ does contribute a high enough power of 2, $2^{i+1}(2^k-1)$ will be a multiple of $m$. Therefore $\rho(m)$ equals one less than the highest power of 2 in $m$.
  \subsubsection{Wait a Minute!}
  The result in the previous section implies that for all odd numbers, the tail will be $-1$. While at first blush this might seem off, as it turns out, this is actually a pretty sensible answer. If we plug in $i = -1$ into our definition of $\rho$, we see it would imply that:
  \begin{align}
    &\forall m, \exists k \in \mathbb{N}, 2m+1 \mid 2^{-1+1}(2^k - 1)\\
    &\forall m, \exists k \in \mathbb{N}, 2^k \equiv 1 \mod 2m+1 \label{necc}
  \end{align}
  But \eqref{necc} is equivalent to \eqref{pf} which we already proved in the last section. Thus, we can see that the $-1$ is purely a consequence of the $(2^n)$ sequence starting with $2^1$ rather than $2^0$. This property of odd moduli will be very useful later on.
  \subsection{Part C: Determining the Period}
  \subsubsection{Examples of Periods}
  \begin{itemize}
  \item $o(3) = 2$ : $(2, 1, 2, 1, 2 ...)$
  \item $o(6) = 2$ : $(2, 4, 2, 4, 2 ...)$
  \item $o(12)= 2$ : $(2, 4, 8, 4, 8 ...)$
  \item It seems like $o(2m) = o(m)$
  \item Additionally the repeating terms of $o(2m)$ are double the repeating terms of $o(m)$.
  \end{itemize}
  \subsubsection{Heuristics for Period Finding}
  One way to formalize what we mean by the period, is that it is the smallest positive integer $x$, such that
  \begin{equation}\label{perdef}
    \forall a, b \in \mathbb{N}, 2^a \equiv 2^{a+bx} \mod m, a > \rho(m)
  \end{equation}
  In practice, we don't need to prove equation \eqref{perdef} holds for all $(a,b)$, because once we prove if for a single $(a,b)$, the rest follows due to the periodic nature of the $(2^n)$ sequence. Additionally, there are actually infinite solutions to \eqref{perdef}, for any solution $x$, all multiples of $x$, will also satisfy the equation. However, only the smallest solution to the equation is the period.\\
  
  This means that even if we can prove an equation of the form,
  \begin{equation}\label{a}
    2^i \equiv 2^{i+j} \mod m, i > \rho(m)
  \end{equation}
  We still won't know that $j$ is the period of $m$, only that $j$ is some multiple of the actual period.\\
  
  The period of any multiple of $m$ must be a multiple of the period of $m$, formally:
  \begin{equation}\label{b}
    o(m) = n \implies o(im) = jn \mid \forall i, \exists j \in \mathbb{N}
  \end{equation}
  To prove \eqref{b}, let $o(im) = r$. Then plugging into \eqref{perdef}:
  \begin{equation}
    \forall a, b \in \mathbb{N}, 2^a \equiv 2^{a+br} \mod im, a > \rho(m)
  \end{equation}
  If the above equation, is true mod $im$, it is also true mod $m$, if we also plug in $b=1$, we get that
  \begin{equation}
    \forall a \in \mathbb{N}, 2^a \equiv 2^{a+r} \mod m, a > \rho(m)
  \end{equation}
  Compare this equation to what we get when we plug $m$ and $n$ into \eqref{perdef}:
  \begin{equation}
    \forall a, b \in \mathbb{N}, 2^a \equiv 2^{a+bn} \mod m, a > \rho(m)
  \end{equation}
  So $r$ must be equal to some $bn$ and we are done.\\

  A lot of the equations above depend on some number being $> \rho(m)$. An easy way to deal with restriction is to note that $\rho(m)$ is cannot be larger than $m$, because there are only $m$ unique elements in $\mathbb{Z}_m$. Alternatively, we can try to use an odd value for $m$, where $\rho(m) = -1$.
  \subsubsection{Proof of $o(2m) = o(m)$}
  We know that $2^{2m+o(m)} \equiv 2^{2m} \mod m$. So, if we multiply both sides by two, we get that $2^{2m + o(m) + 1}\equiv 2^{2m+1} \mod 2m$, therefore by \eqref{a}, $o(m)$ is a multiple of $o(2m)$, and by \eqref{b}, $o(2m)$ is a multiple of $o(m)$, and putting these together we can conclude that $o(2m) = o(m)$.
  \subsubsection{More Examples of Periods}
  \begin{itemize}
  \item $o(3) = 2$ : $(2, 1, 2, 1, 2 ...)$
  \item $o(9) = 6$ : $(2, 4, 8, 7, 5, 1, 2, 4, 8 ...)$
  \item $o(27) = 18$ : $(2, 4, 8, 16, 5, 10, 20, 13, 26, 25, 23, 19, 11, 22, 17, 7, 14, 1, 2, 4, 8 ...)$
  \item Extrapolating, it seems like $o(3m) = 3o(m)$
  \item (Editing: the above seems to only be true for numbers with factors that have large powers of 3)
  \item $o(7) = 3$ and $o(21) = 6$ and $o(63) = 6$ but $o(189) = 18$ and $o(567) = 54$
  \item 3 cases observed: $o(3m) = o(m)$, $o(3m) = 2o(m)$, or $o(3m) = 3o(m)$
  \end{itemize}
  \subsubsection{Proof for  $o(3m)$ if $3 \nmid m$}
  Let $3 \nmid m$. Then, plugging in $a = 3m$, $b = 2$ into \eqref{perdef}:
  \begin{equation}
    m \mid 2^{3m+2o(m)}-2^{3m}
  \end{equation}
  We can also plug in $a = 3m$, $b = o(m)$ into \eqref{perdef} for $\mathbb{Z}_3$.\footnote{recall that $o(3) = 2$}
  \begin{equation}
    3 \mid 2^{3m+o(m)*2}-2^{3m} 
  \end{equation}
  Because $n$ is relatively prime to 3, we know the left side of the equation also is divisible by $3m$:
  \begin{equation}
    3m \mid 2^{3m+2o(m)} - 2^{3m}
  \end{equation}
  Applying \eqref{a} to this, we can conclude $2o(m)$ is a multiple of $o(3m)$. Applying \eqref{b}, we also know that $o(3m)$ is a multiple of $o(m)$. Thus, $o(3m)$ is either $2o(m)$ or $o(m)$. To distinguish between these cases, we must consider if:
  \begin{equation}\label{om}
    3 \mid 2^{3m + o(m)} - 2^{3m}
  \end{equation}
  Which, similarly combines with:
  \begin{equation}
    m \mid 2^{3m + o(m)} - 2^{3m}
  \end{equation}
  to imply:
  \begin{align}
    3m \mid 2^{3m + o(m)} - 2^{3m} \\
    o(m) = ko(3m),\  k \in \mathbb{N}
  \end{align}
  \eqref{om} is only true if $o(m)$ is even\footnote{it is a solution to \eqref{perdef} where $a = 3m$ and $b = 1$}. So if $o(m)$ is even, $o(3m) = o(m)$, otherwise $o(3m) = 2o(m)$.
  \subsubsection{Proof for $o(m)$ if $3 \mid m$}
  We want to find the smallest positive integer $k$ such that $3m \mid 2^{3m + ko(m)} - 2^{3m}$. So:
  \begin{equation}
    3 \mid \frac{2^{3m + ko(m)} - 2^{3m}}{m}
  \end{equation}
   The we divide by $2^{3m}$,\footnote{This is allowed, because $2^{3m}$ is relatively prime to 3} to get that:
  \begin{equation}
    3 \mid \frac{2^{ko(m)}-1}{m}
  \end{equation}
  More intuitively, the above equation implies that the power of three in the numerator $2^{ko(m)}-1$, should be at least 1 greater than the power of three in the denomintaor $m$. So, let us consider the power of the factor of 3 for all possible $a$ in $2^a-1$.\\
  \begin{tabular} {|c|c|c|c|c|c|c|c|}
    \hline $a$                             & 0 & 1 & 2 & 3 & 4  &  5 & 6 \\
    \hline $2^a-1$                         & 0 & 1 & 3 & 7 & 15 & 31 & 63\\
    \hline Highest power of 3 in $2^a-1$   & $\infty$ & 0 & 1 & 0 & 1  & 0  & 2 \\
    \hline 
  \end{tabular}\\
  
  It seems like we found another periodic pattern! Whenever you go from the $n$th to the $n+2$th term in the sequence, $2^{n+2} - 2^n$ is added, which is always equal to 0 modulo 3, as $o(3) = 2$. Additionally every $3^1*2$ terms, $2^a-1$ is divisible by $3^2$, as a number divisible by 3 has been added a multiple of 3 times to 0. Every $3^2*2$ terms, the $2^a-1$ is divisible by $3^3$, as a number divisible by 3 has been added a multiple of $3^2$ times to 0. So in general, for all postive even $a$, the highest power of three in $2^a - 1$ is one more than the highest power of 3 in $\frac{a}{2}$.\\

  Thus, if $3 \mid \frac{\frac{o(m)}{2} * 3}{m}$, then $3m \mid 2^{o(m)}-1$, so $3m \mid 2^{3m + o(m)}-2^{3m}$, so $o(m)$ is a multiple of $o(3m)$. Taking into account the fact that $o(3m)$ is a multiple of $o(m)$, we find that $o(m)= o(3m)$.\\

  If instead $3\nmid \frac{\frac{o(m)}{2} * 3}{m}$, just multiply both sides by three to get $3 \mid \frac{\frac{3o(m)}{2} * 3}{m}$. So instead $3m \mid 2^{3o(m)} - 1$, so $3m \mid 2^{3m+3o(m)} - 2^{3m}$. So $3o(m)$ is a multiple of $o(3m)$, but $o(3m)$ is also a multiple of $o(m)$, but also is not $o(m)$ (because otherwise the we could run the logic in the previous paragraph in reverse and reach a contradiction), which leaves only the solution $o(3m) = 3o(m)$.
  \subsubsection{Summary of $o(3m)$}
  $o(3m) = ko(m)$, where $k$ is equal to:
  \begin{itemize}
  \item $1$, If $m$ isn't divisible by 3 and $o(m)$ is even
  \item $2$, If $m$ isn't divisible by 3 and $o(m)$ is odd
  \item $1$, If $m$ is divisible by 3 and $\frac{\frac{o(m)}{2} * 3}{m}$ is divisible by 3
  \item $3$, If $m$ is divisible by 3 and $\frac{\frac{o(m)}{2} * 3}{m}$ isn't divisible by 3
  \end{itemize}
  \subsection{Generalized Period Finding}
  \subsubsection{Generalizing Prior Techniques}
  Let's try and generalize the table we made in proving $o(3m)$. We start by noting that because the tail of all odd $\mathbb{Z}_m$ is $-1$, we can plug in $a=0$, in  \eqref{perdef} to get:
  \begin{align}
    \forall b \in \mathbb{N}, 2^{0} \equiv 2^{bo(m)} \mod m\\
    2^0 \equiv 2^{o(m)} \equiv 2^{2o(m)}\equiv 2^{3o(m)} \equiv ... \mod m\\
    \forall k \in \mathbb{N}, 2^{(k+1)o(m)} \equiv 2^{ko(m)} \mod m\\
    m \mid 2^{(k+1)o(m)} - 2^{ko(m)} \label{weak}
  \end{align}
  \eqref{weak} is useful for proving the much more general fact that:
  \begin{equation}\label{succ}
    m^{i+1} \mid 2^{(k+1)m^io(m)} - 2^{km^io(m)},\  \forall i, k \in \mathbb{N}
  \end{equation}
  Note that \eqref{weak} is the base case $i=0$ of \eqref{succ}. Using the inductive assumption:
  \begin{align}
    m^{i+1} \mid 2^{(k+1)m^io(m)} - 2^{km^io(m)}\\
    \exists t \in \mathbb{N}, 2^{(k+1)m^io(m)} - 2^{km^io(m)} &= m^{i+1}t\\
    \sum^{km+m-1}_{j=km}2^{(j+1)m^io(m)} - 2^{jm^io(m)} &= \sum^{km+m-1}_{j=km}m^{i+1}t\\
    2^{(km+m)m^io(m)} - 2^{km*m^{i}o(m)} &= m * m^{i+1}t\\
    m^{i+2} \mid 2^{(k+1)m^{i+1}o(m)} - 2^{km^{i+1}o(m)} 
  \end{align}
  And we have proved \eqref{succ}.
  \subsubsection{Another Induction Proof}
  Then, we can use \eqref{succ}, to prove that:
  \begin{equation}\label{powerrule}
    \forall i, m ,p \in \mathbb{N}, 2 \nmid m \implies m^{i+1} \mid 2^{pm^{i}o(m)} - 1
  \end{equation}
  To start, note that base case of $p=0$, works, because $2^0 - 1 = 0$, which is divisible by all positive integers. Then starting from the inductive assumption:
  \begin{equation}\label{assume}
    m^{i+1} \mid 2^{pm^io(m)} - 1 
  \end{equation}
  add \eqref{succ} (with $p$ plugged in for $k$) to get:
  \begin{align}
    \eqref{assume} + \eqref{succ} &= 2^{(p+1)m^io(m)} - 2^{pm^io(m)} +2^{pm^io(m)} - 1\\
    &= 2^{(p+1)m^io(m)} - 1 \label{pushed}
  \end{align}
  Since both \eqref{assume} and \eqref{succ} are divisible by $m^{i+1}$, so does their sum, \eqref{pushed}, so we have proved \eqref{powerrule} by induction.
  \subsubsection{The Bigger Picture}
  The reason that \eqref{powerrule} is useful, is that it allows us to get $o(m^i)$, for any odd $m$ as long as we know $o(m)$. We can quickly plug in $p = 1$ and rearrange it to get:
  \begin{equation}
    2^0 \equiv 2^{0+m^io(m)} \mod m^{i+1}
  \end{equation}
  Using \eqref{a}, we can conclude that $m^io(m)$ is a multiple of $o(m^{i+1})$. To prove instead that:
  \begin{equation}
    \forall m, i \in \mathbb{N}, m \nmid 2 \implies o(m^{i+1}) = m^io(m) 
  \end{equation}
  We additionally need to prove that $m^i$ is the \emph{smallest} coefficient $k$ such that:
  \begin{equation}
    2^0 \equiv 2^{0 + ko(m)} \mod m^{i+1}
  \end{equation}
  It is easy to see that this is true for $i=0$, because:
  \begin{equation}
    2^0 \equiv 2^{0 + m^0o(m)} \mod m^{0+1}
  \end{equation}
  is a special case of the definition of $o(m)$, \eqref{perdef}, where $a=0$ and $b=1$. Then we do induction yet again. If we assume that $m^i$ is the smallest coefficient here:
  \begin{align}
    2^0 &\equiv 2^{0 + m^io(m)} \mod m^{i+1}\\
    m^{i+1} &\mid 2^{m^io(m)} - 1 \\
    m^{i+1} &\mid (2^{m^io(m)})^k(2^{m^io(m)} - 1), \forall k \in \mathbb{N}\\
    2^{m^io(m)(k+1)} - 2^{m^io(m)k} &=  m^{i+1}t, \exists t \in \mathbb{N}\\
    \sum^{m-1}_{k=0}2^{m^io(m)(k+1)} - 2^{m^io(m)k} &= \sum^{m-1}_{k=0}m^{i+1}t\label{why}\\
    2^{m^{i+1}o(m)} - 2^{m^io(m)0} &= m(m^{i+1}t)\\
    m^{i+2} &\mid 2^{m^{i+1}o(m)} - 2^0\\
    2^0 &\equiv 2^{m^{i+1}o(m)} \mod m^{i+2}
  \end{align}
  \eqref{why} is optimal, because numbers divisible by $m^{i+1}$ must be added at least $m$ times\footnote{Technically, the original numbers might also be divisible by $m^{i+2}$, but in that case too many terms were added in previous steps.} to themselves to get a number divisible by $m^{i+2}$, and the left side is the only way to ensure that the terms remain in the form $2^a-1$ after being added. So, $m^{i+1}$ is also the smallest coefficient for $m^{i+2}$, and we are done.\\
  
  This result allows us to find the period of any power of any (odd) prime number with known period. Then we just need some way to multiply together these powers, and we can find the period of any composite number as well\footnote{Powers of 2 can be added without consequence, because $o(2m) = o(m)$}.
  \subsubsection{Motivating Examples}
  \begin{itemize}
  \item $o(5) = 4$, $o(7) = 3$, and $o(35) = 12$
  \item $o(m)o(n) = o(mn)$?
  \item only if $\gcd(m, n) = 1$?
  \item Counterexample: $o(3) = 2$, $o(11) = 10$, $o(33) = 10$
  \item Maybe instead $\lcm(o(m), o(n))$? 
  \end{itemize}
  \subsubsection{Proof of $o(mn) = \lcm(o(m), o(n))$}
  Because we can freely multiply by 2, we only need to consider numbers that are odd, which are more well behaved. The product of two odd numbers is odd, so $\rho(mn) = -1$, so $o(mn)$ is the smallest positive integer satisfying:
  \begin{equation}\label{start}
    2^0 \equiv 2^{o(mn)} \mod mn
  \end{equation}
  If we further assume that $m$ and $n$ are relatively prime (which we can still do if we make $m$ and $n$ powers of different prime numbers). We can conclude:
  \begin{align}
    2^0 \equiv 2^{o(mn)} \mod m \land 2^0 \equiv 2^{o(mn)} \mod n
  \end{align}
  But we also know, given that $m$ and $n$ are also odd, that:
  \begin{align}
    2^0 &\equiv 2^{b_mo(m)} \mod m, \forall b_m \in \mathbb{N}\\
    2^0 &\equiv 2^{b_no(n)} \mod n, \forall b_n \in \mathbb{N}
  \end{align}
  So we find that any multiple of $o(m)$ and $o(n)$ will satsify \eqref{start}, and the least common multiple is the actual period, o(mn).
  \subsubsection{Putting It All Together}
  So given some postive integer $m$ with prime factorization:
  \begin{equation}
    2^x\prod^n_{i=0}p_i^{x_i}
  \end{equation}
  We can conclude that:
  \begin{equation}
    o(m) = \lcm(p_0^{(x_0-1)}o(p_0), p_1^{(x_1-1)}o(p_1), \dots, p_n^{(x_n-1)}o(p_n))
  \end{equation}
  To find the period of any odd prime, we can again use the odd number shortcut: just compute the smallest positive integer $k$ such that
  \begin{equation}
    2^k \equiv 1 \mod p_i
  \end{equation}
  \section{Problem 2: Algebraic Properties}
  \subsection{Properties Summary}
  Given some set $S$ over which $+$ and $\times$ are defined it satisifies:
  \begin{itemize}
  \item Property $i$ iff $\nexists a \in S$ such that $a \times a = 1$ and $a \neq \pm 1$
  \item Property $ii$ iff $\nexists x \in S$ such that $2x = 0$ and $x \neq 0$
  \item Property $iii$ iff $\nexists c \in S$ such that $c \times c = 0$ and $c \neq 0$
  \end{itemize}
  \subsection{Examples of Sets}
  \begin{itemize}
  \item $\mathbb{Z}$ has properties $i$, $ii$, and $iii$\\
    Given that $\forall n \in \mathbb{Z}$, $n$ is also $\in \mathbb{Q}$, and that $i+j=k$ and $ij=l$ over both $\mathbb{Q}$ and $\mathbb{Z}$, $\forall i, j \in \mathbb{Z}$, we can conclude that there are no counterexamples $a, x, c \in \mathbb{Z}$ as long as there are no counterexamples in $\mathbb{Q}$. Therefore, it is sufficient to prove properties $i$, $ii$, and $iii$ hold in $\mathbb{Q}$. 
  \item $\mathbb{Q}$ has properties $i$, $ii$, and $iii$
    \begin{enumerate}
    \item if $a^2 = 1$ then $a^2-1 = 0$. If we solve this, we can conclude that $a$ must be $1$ or $-1$. We know that these are the only two solutions, because $a^2-1$ has degree 2. Thus, property $i$ holds for $\mathbb{Q}$.
    \item Similarly, $2x = 0$ must have 1 solution, given that it is a 1st degree polynomial. That solution is clearly $x = 0$. Thus, property $ii$ holds for $\mathbb{Q}$.
    \item For $c^2=0$, we see that the solution is $c=0$ with multiplicity 2, which makes it the only solution. Thus, property $iii$ holds for $\mathbb{Q}$.
    \end{enumerate}
  \item $4\mathbb{Z}$ has properties $i$, $ii$, and $iii$\\
    We can use similar logic as $\mathbb{Z}$ given that $\forall n \in 4\mathbb{Z}$, $n$ is also $\in \mathbb{Q}$, and that $i+j=k$ and $ij=l$ over both $\mathbb{Q}$ and $4\mathbb{Z}$, $\forall i, j \in 4\mathbb{Z}$.
  \item $\mathbb{Z}_3$ has properties $i$, $ii$, and $iii$\\
    \begin{tabular} {|c|c|c|c|}
      \hline $n$   & 0 & 1 & 2\\
      \hline $2n$  & 0 & 2 & 1\\
      \hline $n^2$ & 0 & 1 & 1\\                       
      \hline 
    \end{tabular}\\
  \item $\mathbb{Z}_8$ has none of $i$, $ii$, or $iii$\\
    \begin{tabular} {|c|c|c|c|c|c|c|c|c|}
      \hline $n$   & 0 & 1 & 2 & 3 & 4 & 5 & 6 & 7\\ 
      \hline $2n$  & 0 & 2 & 4 & 6 & 0 & 2 & 4 & 6\\
      \hline $n^2$ & 0 & 1 & 4 & 1 & 0 & 1 & 4 & 1\\                       
      \hline 
    \end{tabular}\\
    \begin{itemize}
    \item $a=3,5$ are counterexamples for property $i$
    \item $x=4$ is a counterexample for property $ii$
    \item $c=4$ is a counterexample for property $iii$
    \end{itemize}
  \item $\mathbb{Z}_9$ has property $i$ and $ii$, but not $iii$\\
    \begin{tabular} {|c|c|c|c|c|c|c|c|c|c|}
      \hline $n$   & 0 & 1 & 2 & 3 & 4 & 5 & 6 & 7 & 8\\
      \hline $2n$  & 0 & 2 & 4 & 6 & 8 & 1 & 3 & 5 & 7\\
      \hline $n^2$ & 0 & 1 & 4 & 0 & 7 & 7 & 0 & 4 & 1\\                   
      \hline 
    \end{tabular}\\
    \begin{itemize}
    \item $c=3,6$ are counterexamples for property $iii$.
    \end{itemize}
  \item $4\mathbb{Z}_{12}$ has properties $i$, $ii$, and $iii$\\
    \begin{tabular} {|c|c|c|c|}
      \hline $n$   & 0 & 4 & 8\\
      \hline $2n$  & 0 & 8 & 4\\
      \hline $n^2$ & 0 & 4 & 4\\                       
      \hline 
    \end{tabular}\\
  \item $\mathbb{Z}_{13}$ has property $i$ and $ii$, and $iii$\\
    \begin{tabular} {|c|c|c|c|c|c|c|c|c|c|c|c|c|c|}
      \hline $n$   & 0 & 1 & 2 & 3 & 4 & 5 & 6 & 7 & 8 & 9 & 10 & 11 & 12\\
      \hline $2n$  & 0 & 2 & 4 & 6 & 8 & 10 & 12 & 1 & 3 & 5 & 7 & 9 & 11\\
      \hline $n^2$ & 0 & 1 & 4 & 9 & 3 & 12 & 10 & 10 & 12 & 3 & 9 & 4 & 1\\                   
      \hline 
    \end{tabular}\\
  \end{itemize}
  \subsection{Similarity}
  $\mathbb{Q}$, $\mathbb{Z}$, and $4\mathbb{Z}$ all have $+$ and $\times$ defined the same way, and have properties $i$, $ii$, and $iii$.
  Of the $\mathbb{Z}_m$'s, I notice that $4\mathbb{Z}_{12}$ was just $\mathbb{Z}_3$, but with every element multiplied by 4. $\mathbb{Z}_{13}$ was the other number system with all three properites, probably because it was also prime.
  $\mathbb{Z}_9$ and $\mathbb{Z}_8$ were the odd one's out. I would put the former in a category of odd composites or maybe odd squares, and the latter in a category of even numbers.
  \section{Problem 3: Rossie the Robot}
  \subsection{Part A: Returning to the Start}
  \subsubsection{A Toy Example}
  With $SRSRSRSRSR$ and $\theta = 4\pi/5$, Rossie traces out a five pointed star.
  \subsubsection{Generalizing}
  Define $SR_n$ to be the sequence $SRSR...SR$ where $SR$ is repeated n times. $\forall p, q \in \mathbb{Z}$ and $\gcd(p,q) = 1$, if $\theta = \frac{2\pi p}{q}$, Rossie can return to Start by following the sequence $SR_q$ \footnote{This is only well-defined if $q$ is a positive integer. However, if $q$ is not, we can simply multiply both $p$ and $q$ by $-1$ to get an equivalent $\theta$.}.
  \subsubsection{Proof}
   If we map Rossie's movements on the complex plane with O at 0, we can see that before the $m$th $S$, Rossie has rotated an angle of $\frac{2\pi p(m-1)}{q}$. Therefore, after $SR_q$, Rossie ends up at:
  \begin{equation}
    \sum^{q}_{m=1}e^{\frac{2\pi p(m-1)i}{q}} = \sum^{q-1}_{m=0}e^{\frac{2\pi pmi}{q}}
  \end{equation}
  We note that the above sum equals a geometric series with initial term $e^{\frac{2\pi p*0*i}{q}} = e^0 = 1$, common ratio $e^{\frac{2\pi pi}{q}}$, and $q$ terms.
  \begin{align}
    \sum^{q-1}_{m=0}e^{\frac{2\pi pmi}{q}} = \frac{1-(e^{\frac{2\pi pi}{q}})^q}{1-e^{\frac{2\pi pi}{q}}}
    = \frac{1-(e^{\pi i})^{2p}}{1 - e^{\frac{2\pi pi}{q}}}
    = \frac{1-(-1)^{2p}}{1 - e^{\frac{2\pi pi}{q}}}
    = 0
  \end{align}
  
  Now that we have proved that following $SR_q$ leads Rossie back to O, we can prove that it also orients Rossie towards the positive x-axis by noting that Rossie must rotate $\frac{2\pi p}{q}$ radians q times, and thus rotates a total of $2\pi p$ radians, or $p$ full revolutions. Therefore, Rossie is also at Start after $SR_q$.
  \subsection{Part B: Returning to O}
    Using the complex plane idea again, we know that the path traced is equal to:
  \begin{align}
    3 + 2e^{\cos^{-1}(-\frac{1}{3})i} + 3e^{2\cos^{-1}(-\frac{1}{3}) i}
  \end{align}
  which we can reduce with the help of Euler's Formula
  \begin{align}
    &= 3 + 2(\cos(\cos^{-1}(-\frac{1}{3}))) + 2(\sin(\cos^{-1}(-\frac{1}{3})))i \\  
    &\hspace{18px} + 3(\cos(2\cos^{-1}(-\frac{1}{3}))) + 3(\sin(2\cos^{-1}(-\frac{1}{3})))i \\
    &= 3 + -\frac{2}{3} + \frac{4\sqrt2i}{3} + 3(2\sin(\cos^{-1}(-\frac{1}{3}))\cos(\cos^{-1}(-\frac{1}{3})))i\\
    &\hspace{20px} + 3(\cos^2(\cos^{-1}(-\frac{1}{3})) - \sin^2(\cos^{-1}(-\frac{1}{3})))\\
    &= \frac{7 + 4\sqrt2i}{3} + 6(\frac{2\sqrt2 * -1}{3*3})i + 3((-\frac{1}{3})^2 - (\frac{2\sqrt2}{3})^2)\\
    &= \frac{7 + 4\sqrt2i - 4\sqrt2i}{3} + 3(\frac{1}{9} - \frac{8}{9})\\
    &= \frac{7 - 7}{3} \\
    &= 0
  \end{align}
  \subsubsection{Returning to Start}
  In order for $\theta$ to have some sequence that returns to Start, it must necessarily after some amount of rotations orient towards the positive x-axis. At least one rotation must be done however, because after the first $S$, Rossie can only move rightwards, despite being to the left of O. In other words, there must $\exists k, l \in \mathbb{N} \mid k, l \neq 0$ such that $k\theta = 2\pi l$ or $\frac{\theta}{2\pi} = \frac{l}{k}$. So, for proving that $\theta = \cos^{-1}(-\frac{1}{3})$ cannot return to Start, it is sufficient to prove that $\frac{\cos^{-1}(-\frac{1}{3})}{2\pi}$ is irrational.
  \subsubsection{``Proving'' Irrationality}
  While I was unable to prove that it is irrational, I'm confident that it is, because of the wording of the next question.\\
  
  In part A, we used a correspondence between rational numbers and the roots of unity. In a similar vein, proving irrationality is equivalent to proving that a complex number is not a root of unity. For the case of $\cos^{-1}(-\frac{1}{3})$, we need to prove:
  \begin{align}
    \forall p, q \in \mathbb{N}, e^{pi\cos^{-1}(-\frac{1}{3})} &\neq e^{2\pi iq}\\
    (-\frac{1}{3} + \frac{2\sqrt2}{3}i )^p &\neq 1
  \end{align}
  \subsection{Part C: Generalizing Part B}
  Given some angle $\theta$ and some sequence with $n$ rotations that brings it to O, there must exist a corresponding polynomial:
  \begin{align}
    \sum^{n-1}_{k=0}a_kz^k = 0
  \end{align}
  where the coefficient $a_k$ is the number of $S$'s in the sequence that happen between the $k$th and $k+1$th $R$ and $z = e^{i\theta}$. So given some set of coefficients $\{a_0, a_1, ..., a_{n-1} \mid a_j \in \mathbb{N}\}$, we can solve the corresponding polynomial to get some roots $\{z_0, z_1, ..., z_{n-1}\}$ Then from each of these we can get an angle $\theta_j$\footnote{that is between 0 and $2\pi$}:
  \begin{align}
    z_j = e^{i\theta_j}\\
    \theta_j = \frac{\ln(z_j)}{i}
  \end{align}
  After that, repeat over all $ n \in \mathbb{N}$ to get all $\theta$'s that return to O. Futhermore, if $\frac{\theta}{2\pi}$ is irrational, then $\theta$ doesn't return to Start.\\
  
  We can also apply this to note that the proof in Part A is a special case where all the coeeficients $a_j$ are equal to 1 (or equivalently, all the $a_j$ are equal to each other), the roots of which are the $n$th roots of unity, which add up to zero.
  \subsection{Part D: Not Returning to O}
  Then running the logic in Part C in reverse, if $\exists z \in \mathbb{C}$ such that there is no polynomial with non-negative integer coefficients that has $\frac{z}{\lvert z \rvert}$ as a root, then if Rossie has angle $\frac{\ln(\frac{z}{\lvert z \rvert})}{i}$, Rossie can never return to O.
  \subsubsection{Proving Existence}
  We can think of the operation in part C as a function $f$ that maps a set of $n$ $a_j$ to a set of $n$ $\theta_j$\footnote{Again, only considering angles between 0 and $2\pi$}, $\forall n \in \mathbb{N}$. Consider the domain of $f$, $D$. $D$ has cardinality:
  \begin{align}
    \sum^{\aleph_0}_{k=0}\aleph_0^k = \aleph_0
  \end{align}
  Then, the range of $f$, $R$ also has cardinality $\leq \aleph_0$. Additionally, each element in $R$ has a maximum of $\aleph_0$ elements, so an upper bound on the cardinality of the union $U$ of the elements of $R$, is
  \begin{align}
    \rvert U \lvert \  \leq \aleph_0 \times \aleph_0 = \aleph_0
  \end{align}
  But $U$ is also the set of all angles $\theta$ which return to O after a postive integer number of steps. But there are only at most a countably infinite number of angles in $U$, while there are an uncountably infinite number of possible angles Rossie could have. Therefore there are an uncountably infinite number of angles which can never return to O.
  \section{Appendix}
  \subsection{Programmatic Generation of Periods}
  Here is a short python script I used to generate the tables below:
  \lstinputlisting[language=Python]{ross.py}
  \subsection{Table of Periods}
  \csvautotabular{periods1.csv}
  \csvautotabular{periods2.csv}
  \csvautotabular{periods3.csv}
  \csvautotabular{periods4.csv}
  \csvautotabular{periods5.csv}
  \csvautotabular{periods6.csv}
\end{document}